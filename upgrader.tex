%!TEX root = main.tex

\section{Fluidized bed VPU}

The vapor phase upgrader (VPU) catalytically upgrades pyrolysis vapors from the pyrolyzer unit. The diagram shown Figure \ref{fig:upgrader} displays the process flows and basic geometry of the fluidized bed VPU system. Heat jackets and insulation around the VPU are used to maintain temperature within the system. All dimensions were taken from the technical drawing M\_FTLB\_155-01 and from email correspondence with Rick French.

\begin{figure}[H]
    \centering
    \includegraphics[width=\textwidth]{upgrader}
    \caption{Fluidized bed vapor phase upgrader.}
    \label{fig:upgrader}
\end{figure}

The inlet gas at the bottom of the VPU is a combination of pyrolysis vapors and nitrogen gas. Zeolite catalyst is used as the bed material in the VPU. Properties of the catalyst as measured by NETL are available in the table below.

\begin{table}[H]
    \centering
    \caption{Gas composition at the inlet and outlet.}
    \label{tab:gas-comp}
    \begin{tabular}{lrr}
        \toprule
        Component & Inlet & Outlet \\
        \midrule
        N$_2$               & 70-80\%   & 70\%  \\
        CO$_2$              & 1\%       & 3\%   \\
        CO                  & 2\%       & 6\%   \\
        H$_2$O              & 4\%       & 7\%   \\
        Phenols             & 1\%       & 1\%   \\
        Methoxy Phenols     & 10\%      & 1\%   \\
        Carboxylic Acids    & 2\%       &       \\
        Carbonyls           & 5\%       &       \\
        Levoglucosan        & 2\%       &       \\
        Light Olefins       &           & 1\%   \\
        Aromatics           &           & 4\%   \\
        Indenols, Naphthols &           & 1\%   \\
        \bottomrule
    \end{tabular}
\end{table}

An HZSM-5 catalyst is the bed material for the vapor phase upgrader. Particle size ranges from 300 to 1000 um. Coking effect is typically 10\%. The silica to alumina ratio (SAR) is 30:1. Residence time in the reactor is generally 14 minutes. Silca binder SiO$_2$ is 20\% by weight, acid site concentration 960 umol/g and BET surface area of 370 m$^2$/g.

Various technical documents and journal articles about the 2FBR system are listed below.

\begin{enumerate}
    \item Daniel Carpenter, Steve Deutch, Anne Starace. NREL Milestone Completion Report. Thermochemical Feedstock Interface, WBS 2.9.1.3, June 2014.
    \item Richard French. Modeling of a bench-scale biomass pyrolyzer: An experimentalist’s viewpoint. 249th American Chemical Society National Meeting and Exposition, Denver, Colorado, March 2015.
    \item Daniel Howe, Tyler Westover, Daniel Carpenter, Daniel Santosa, Rachel Emerson, Steve Deutch, Anne Starace, Igor Kutnyakov, and Craig Lukins. Field-to-Fuel Performance Testing of Lignocellulosic Feedstocks: An Integrated Study of the Fast Pyrolysis-Hydrotreating Pathway. Energy \& Fuels, 2015, 29, 3188-3197.
    \item Anna Trendewicz, Robert Evans, Abhijit Dutta, Robert Sykes, Daniel Carpenter, Robert Braun. Evaluating the effect of potassium on cellulose pyrolysis reaction kinetics. Biomass and Bioenergy, 2015, 74, 15-25.
    \item Matthew Yung, Alexander Stanton, Kristiina Iisa, Richard French, Kellene Orton, Kimberly Magrini. Multiscale Evaluation of Catalytic Upgrading of Biomass Pyrolysis Vapors on Ni- and Ga-Modified ZSM-5. Energy and Fuels, 2016, 30, 9471-9479.
    \item Daniel Carpenter, Tyler Westover, Daniel Howe, Steve Deutch, Anne Starace, Rachel Emerson, Sergio Hernandez, Daniel Santosa, Craig Lukins, Igor Kutnyakov. Catalytic hydroprocessing of fast pyrolysis oils: Impact of biomass feedstock on process efficiency. Biomass and Bioenergy, 2017, 96, 142-151.
    \item Kristiina Iisa, Richard J. French, Kellene A. Orton, Abhijit Dutta, Joshua A. Schaidle. Production of low-oxygen bio-oil via ex situ catalytic fast pyrolysis and hydrotreating. Fuel 2017, 207, 413-422.
\end{enumerate}
