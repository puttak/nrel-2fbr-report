%!TEX root = main.tex

\section{Fluidized bed pyrolyzer}

The biomass thermochemical conversion unit in the NREL 2FBR system is a two-inch diameter bubbling fluidized bed (BFB) reactor that operates at fast pyrolysis conditions. An overview of the pyrolyzer components along with its process flows are shown in Figure \ref{fig:pyrolyzer}. See the following subsections for more information on reactor dimensions, operating conditions, feedstock characteristics, bed material properties, and biomass pyrolysis kinetic schemes.

\begin{figure}[H]
    \centering
    \includegraphics[width=\textwidth]{pyrolyzer}
    \caption{Bubbling fluidized bed biomass pyrolysis reactor in the NREL 2FBR system. Reactor components (left) along with inlet and outlet process flows (right).}
    \label{fig:pyrolyzer}
\end{figure}

\subsection{Dimensions and operating conditions}

Dimensions for the reactor tube, feed inlet, insulation, heat jacket, and distributor plate are given in Table \ref{tab:dimensions}. The main reactor tube is a 2-inch Schedule 40 pipe; therefore, the inner and outer reactor diameters are determined from nominal pipe size tables. The gas distributor contains 18 holes in a triangular pattern.

Typical operating conditions of the pyrolyzer are presented in Table \ref{tab:operating}. Pressure drop across the distributor is about 80-90 inches of H$_2$O. Nitrogen gas is used to fluidize the bed and assist biomass particles through the feed inlet tube. Experiments are conducted with an initial mass of sand in the bed; sand is not fed into the reactor during operation. Insulation surrounds the reactor while heat jackets extend almost the entire height of the unit. A cooling jacket surrounds the feed inlet tube. Pyrolysis vapors exit directly out the top of the reactor via a straight tube.

\begin{table}[H]
    \centering
    \caption{Dimensions for components of the fluidized bed pyrolysis reactor. Values from NREL \cite{French-2019}.}
    \label{tab:dimensions}
    \begin{tabular}{llrll}
        \toprule
        Component & Dimension & Value & Units & Description \\
        \midrule
        Reactor
            & d$_\textrm{inner}$ & 5.25 & cm & inner diameter \\
            & d$_\textrm{outer}$ & 6.03 & cm & outer diameter \\
            & h$_\textrm{static}$ & 10.16 & cm & estimated static bed height \\
            & h$_\textrm{total}$ & 43.18 & cm & total height of reactor tube \\
        Feed inlet
            & d$_\textrm{inner}$ & 1.27 & cm & inner diameter \\
            & h$_\textrm{feed}$ & 1.9 & cm & height from top of distributor \\
            & len & 18.29 & cm & length of feed inlet tube \\
        Insulation
            & h$_\textrm{insul}$ & 43.18 & cm & insulation height \\
            & th$_\textrm{insul}$ & 10 & cm & thickness \\
        Heat jacket
            & h$_\textrm{jack}$ & 35 & cm & height of jacket \\
            & th$_\textrm{jack}$ & 5 & cm & thickness of jacket \\
        Distributor
            & d$_\textrm{orif}$ & 0.08 & cm & diameter of distributor orifices \\
            & n & 18 & - & number of orifices \\
            & th$_\textrm{dist}$ & 3.17 & mm & thickness of distributor \\
        \bottomrule
    \end{tabular}
\end{table}

\begin{table}[H]
    \centering
    \caption{Typical operating conditions for the fluidized bed pyrolysis reactor. Values from NREL \cite{French-2019}.}
    \label{tab:operating}
    \begin{tabular}{llrll}
        \toprule
        Component & Condition & Value & Units & Description \\
        \midrule
        Reactor
            & p$_\textrm{abs}$ & 101.3 & kPa & absolute pressure in reactor \\
            & p$_\textrm{atm}$ & 81 & kPa & atmospheric pressure \\
            & tk$_\textrm{amb}$ & 300.15 & K & ambient air temperature \\
        Bed
            & p$_\textrm{bed}$ & 115 & kPa & absolute bed pressure \\
            & tk$_\textrm{bed}$ & 773.15 & K & bed temperature \\
        Inlet gas
            & $\Delta$p & 21.17 & kPa & distributor pressure drop \\
            & p$_\textrm{gas}$ & 110 -- 140 & kPa & absolute inlet gas pressure \\
            & tk$_\textrm{gas}$ & 773.15 & K & inlet gas temperature \\
            & u$_\textrm{gas}$ & 14 & SLM & nitrogen gas flowrate, 0.292 g/s \\
        Outlet gas
            & p$_\textrm{gas}$ & 90 -- 110 & kPa & absolute outlet gas pressure \\
        Secondary gas
            & tk$_\textrm{gas}$ & 298.15 & K & gas temperature \\
            & u$_\textrm{gas}$ & 1.4 & SLM & nitrogen gas flowrate, 0.0292 g/s \\
        Biomass feed
            & $\dot{\textrm{m}}_\textrm{feed}$ & 420 & g/hr & biomass feedrate \\
            & tk$_\textrm{feed}$ & 298.15 & K & biomass temperature \\
        \bottomrule
    \end{tabular}
\end{table}

\subsection{Feedstock properties}

Several woody feedstocks have been tested in the NREL pyrolysis reactor; however, loblolly pine is the current focus of ongoing experiments. The wood industry refers to this species of wood as southern yellow pine.

Particle diameters are from the cleanpine-particle-size.xlsx spreadsheet provided by NREL.

Particle size 10th percentile 133.2 um, particle size 50th percentile 534.4 um, and particle size 90th percentile 1057.4 um. Surface weighted mean size 248.2 um and volume weighted mean size 573.9 um.

\begin{table}[H]
    \centering
    \caption{Loblolly pine feedstock properties for the NREL 2FBR pyrolyzer. Heat capacity is a function of particle temperature.}
    \label{tab:feedstock}
    \begin{tabular}{lrll}
        \toprule
        Feed Property & Value & Units & Description \\
        \midrule
        C & 49.6 & wt\% & carbon \cite{Iisa-2016} \\
        H & 6.3 & wt\% & hydrogen \cite{Iisa-2016} \\
        O & 43.5 & wt\% & oxygen \cite{Iisa-2016} \\
        N & 0.05 & wt\% & nitrogen \cite{Iisa-2016} \\
        S & 0.12 & wt\% & sulfur \cite{Iisa-2016} \\
        VM & 85.2 & wt\% & volatile matter \cite{Iisa-2016} \\
        FC & 11.5 & wt\% & fixed carbon \cite{Iisa-2016} \\
        MC & 2.9 & wt\% & moisture content \cite{Iisa-2016} \\
        Ash & 0.43 & wt\% & ash \cite{Iisa-2016} \\
        $\rho$ & 540 & kg/m$^3$ & particle density \cite{WoodHandbook-2010} \\
        C$_\textrm{p}$ & $0.1031 + 0.003867\;\textrm{T}$ & kJ/kgK & heat capacity \cite{WoodHandbook-2010} \\
        k & 0.12 & W/mK & thermal conductivity \cite{WoodHandbook-2010} \\
        \bottomrule
    \end{tabular}
\end{table}

\subsection{Bed properties}

Bed material for the pyrolyzer is approximately 330 g of sand from a sieve cut of 0.3 to 0.5 mm. Sand properties for the pyrolyzer bed material are presented in Table \ref{tab:sand-properties}.

\begin{table}[H]
    \centering
    \caption{Bed properties for fluidized sand particles in the pyrolyzer.}
    \label{tab:sand-properties}
    \begin{tabular}{lrll}
        \toprule
        Bed Property & Value & Units & Description \\
        \midrule
        d$_\textrm{p}$ & 0.3 - 0.5 & mm & particle diameter \\
        $\epsilon$ & 0.76 & - & average emissivity of particles \\
        m$_\textrm{bed}$ & 330 & g & mass of bed material \\
        $\phi$ & 0.86 & - & particle sphericity \\
        $\rho_\textrm{app}$ & 2500 & kg/m$^3$ & apparent density \\
        $\rho_\textrm{bulk}$ & 1510 & kg/m$^3$ & bulk density \\
        \bottomrule
    \end{tabular}
\end{table}

\subsection{Insulation and heat jacket properties}

The BFB pyrolyzer is insulated along the height of the reactor.

\begin{table}[H]
    \centering
    \caption{Insulation properties for the pyrolysis reactor.}
    \label{tab:insulation}
    \begin{tabular}{lrll}
        \toprule
        Insulation Property & Value & Units & Description \\
        \midrule
        $\epsilon$ & 0.92 & - & emissivity of insulation surface \\
        k & 0.0022 & W/mK & thermal conductivity \\
        \bottomrule
    \end{tabular}
\end{table}

\begin{table}[H]
    \centering
    \caption{Heat jacket properties for the pyrolysis reactor.}
    \label{tab:heatjacket}
    \begin{tabular}{lrll}
        \toprule
        Heat Jacket Property & Value & Units & Description \\
        \midrule
        q & 0.38 & kW & heat transfer to interior \\
        \bottomrule
    \end{tabular}
\end{table}


