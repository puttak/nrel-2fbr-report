%!TEX root = main.tex

\section{Fluidized bed pyrolyzer}

The biomass pyrolysis unit in the NREL 2FBR system is a two-inch diameter bubbling fluidized bed (BFB) reactor that operates at fast pyrolysis conditions. An overview of the pyrolyzer components along with its process flows are shown in Figure \ref{fig:pyrolyzer}. See the following subsections for more information on reactor dimensions, typical operating conditions, feedstock characteristics, bed material properties, and biomass pyrolysis kinetic schemes.

\begin{figure}[H]
    \centering
    \includegraphics[width=\textwidth]{pyrolyzer}
    \caption{Bubbling fluidized bed biomass pyrolysis reactor in the NREL 2FBR system. Reactor components (left) along with inlet and outlet process flows (right).}
    \label{fig:pyrolyzer}
\end{figure}

\subsection{Dimensions and operating conditions}

The distributor in the pyrolysis reactor contains 18 holes in a triangular pattern. Pressure drop across the distributor is about 80-90 inches of H$_2$O. Nitrogen gas is used to fluidize the bed and feed biomass particles. Experiments are conducted with an initial mass of sand in the bed; sand is not fed into the reactor during operation. Pyrolysis vapors exit directly out the top of the reactor via a straight tube.

\begin{table}[H]
    \centering
    \caption{Dimensions and properties for components of the fluidized bed pyrolysis reactor.}
    \label{tab:dimensions}
    \begin{tabular}{llrl}
        \toprule
        Component & Name & Value & Description \\
        \midrule
        Reactor
            & D$_\textrm{inner}$ & 5.08 cm & inner diameter \\
            & H$_\textrm{static}$ & 10.16 cm & static bed height \\
            & H$_\textrm{total}$ & 43.18 cm & total height of reactor tube \\
        Feed inlet
            & H$_\textrm{feed}$ & 1.9 cm & feed height above distributor \\
            & D$_\textrm{inner}$ & x & inner diameter \\
        Insulation & $\epsilon$ & 0.92 & emissivity of insulation surface \\
            & H$_\textrm{insul}$ & 43.18 cm & insulation height \\
            & k$_\textrm{insul}$ & 0.0022 W/mK & thermal conductivity \\
            & W$_\textrm{insul}$ & 10 cm & thickness \\
        Heat jacket & H$_\textrm{jack}$ & 35 cm & height of jacket \\
            & W$_\textrm{jack}$ & 5 cm & thickness of jacket \\
            & Q & 0.38 kW & heat transfer to interior \\
        Distributor
            & D$_\textrm{orif}$ & 0.79 mm & diameter of distributor orifices \\
            & n & 18 & number of orifices \\
            & W & 3.17 mm & thickness of distributor plate \\
        \bottomrule
    \end{tabular}
\end{table}

\begin{table}[H]
    \centering
    \caption{Typical operating conditions for the fluidized bed pyrolysis reactor.}
    \label{tab:operating}
    \begin{tabular}{llrl}
        \toprule
        Component & Parameter & Value & Description \\
        \midrule
        Reactor
            & P$_\textrm{avg}$ & 101.3 kPa (abs) & average pressure \\
            & P$_\textrm{atm}$ & 81 kPa & atmospheric pressure \\
            & T$_\textrm{amb}$ & 300.15 K & ambient air temperature \\
        Bed material
            & $\epsilon$ & 0.76 & average emissivity of particles \\
            & m$_\textrm{bed}$ & 330 g & mass of sand particles in bed \\
            & $\rho_\textrm{bulk}$ & 1510 kg/m$^3$ & bulk particle density \\
            & $\rho_\textrm{app}$ & 2500 kg/m$^3$ & apparent particle density \\
            & $\phi$ & 0.86 & particle sphericity \\
            & T & 773.15 K & bed temperature \\
            & P & 16.7 psi (115 kPa abs.) & bed pressure \\
        Inlet gas
            & $\Delta$p & 21.17 kPa & distributor pressure drop \\
            & U$_\textrm{gas}$ & 14 SLM (0.292 g/s) & nitrogen gas flowrate \\
            & T$_\textrm{gas}$ & 773.15 K & inlet gas temperature \\
            & P$_\textrm{gas}$ & 19.29 psi [133 kPa (abs)] & inlet gas pressure \\
        Secondary gas
            & U$_\textrm{gas}$ & 1.4 SLM (0.0292 g/s) & nitrogen gas flowrate \\
            & T$_\textrm{gas}$ & 298.15 K & gas temperature \\
        Biomass feed
            & F$_\textrm{feed}$ & 425 g/hr & biomass feedrate \\
            & T$_\textrm{feed}$ & 298.15 K & biomass temperature \\
        \bottomrule
    \end{tabular}
\end{table}

\subsection{Feedstock properties}

Several woody feedstocks have been tested in the pyrolysis unit; however, loblolly pine is the current focus of ongoing experiments. Properties of loblolly pine from the Wood Handbook are provided in Table XX shown below.

Particle diameters are from the cleanpine-particle-size.xlsx spreadsheet provided by NREL.

Loblolly pine heat capacity of pine calculated from $0.1031 + 0.003867 T$ kJ/(kg K). Particle size 10th percentile 133.2 um, particle size 50th percentile 534.4 um, and particle size 90th percentile 1057.4 um. Surface weighted mean size 248.2 um and volume weighted mean size 573.9 um. Ultimate analysize values on a wet basis taken from Iisa 2017 paper are C 49.6\%, H 6.3\%, O 43.5 \%, N 0.1\%, S less than 0.1\%, ash 0.3\%, and water content 2.3\%.

\begin{table}[H]
    \centering
    \caption{Loblolly pine feedstock properties for NREL 2FBR pyrolyzer. Heat capacity is a function of temperature.}
    \label{tab:feedstock}
    \begin{tabular}{lrl}
        \toprule
        Property & Value & Description \\
        \midrule
        C & 50.6 \% & ultimate analysis carbon \\
        H & 6.6 \% & ultimate analysis hydrogen \\
        O & 42.35 \% & ultimate analysis oxygen \\
        N & 0.03 \% & ultimate analysis nitrogen \\
        S & 0.02 \% & ultimate analysis sulfur \\
        ash & 0.4 \% & ultimate analysis ash \\
        VM & 85.2 \% & proximate analysis volatile matter \\
        FC & 11.5 \% & proximate analysis fixed carbon \\
        MC & 2.9 \% & proximate analysis moisture content \\
        ash & 0.4 \% & proximate analysis ash \\
        $\rho$ & 540 kg/m$^3$ & particle density \\
        C$_\textrm{p}$ & $0.1031 + 0.003867\;\textrm{T}$ kJ/kgK & heat capacity \\
        k & 0.12 W/mK & thermal conductivity \\
        \bottomrule
    \end{tabular}
\end{table}

\subsection{Bed properties}

Sand properties for the pyrolyzer bed material are presented in Table \ref{tab:sand-properties}. Particle size as sieved was reported as 0.3 to 0.5 mm.

\begin{table}[H]
    \centering
    \caption{Sand particle properties for the pyrolyzer bed material.}
    \label{tab:sand-properties}
    \begin{tabular}{llrr}
        \toprule
        Property & Description & Value & Units \\
        \midrule
        d$_p$ & diameter & 0.0003 & m \\
        $\phi$ & sphericity & 0.86 & - \\
        $\rho_{app}$ & apparent density & 2500 & kg/m$^3$ \\
        $\rho_{bulk}$ & bulk density & 1510 & kg/m$^3$ \\
        \bottomrule
    \end{tabular}
\end{table}

\subsection{Biomass pyrolysis kinetics}

The Ranzi et al. kinetic scheme for biomass fast pyrolysis is described in this section.

\begin{table}[H]
    \centering
    \caption{Solid species in Ranzi kinetic scheme.}
    \begin{tabular}{@{}clll@{}}
        \toprule
        Item & Abbreviation & Name & Formula \\
        \midrule
        1   & CELL      & cellulose                     & C$_6$H$_{10}$O$_5$ \\
        2   & CELLA     & activated cellulose           & C$_6$H$_{10}$O$_5$ \\
        3   & CHAR      & char                          & C \\
        4   & GCO       & metaplastic carbon monoxide   & CO \\
        5   & GCO2      & metaplastic carbon dioxide    & CO$_2$ \\
        6   & GCH4      & metaplastic methane           & CH$_4$ \\
        7   & GC2H4     & metaplastic ethylene          & C$_2$H$_4$ \\
        8   & GCH3OH    & metaplastic methyl alcohol    & CH$_4$O \\
        9   & GCOH2     & metaplastic formaldehyde      & CH$_2$O \\
        10  & GH2       & metaplastic hydrogen          & H$_2$ \\
        11  & GMSW      & hemicellulose softwood        & C$_5$H$_8$O$_4$ \\
        12  & HCE1      & hemicellulose                 & C$_5$H$_8$O$_4$ \\
        13  & HCE2      & hemicellulose                 & C$_5$H$_8$O$_4$ \\
        9   & HMWL      & heavy molecular weight lignin & C$_{24}$H$_{28}$O$_4$ \\
        14  & LIG       & lignin                        & C$_{11}$H$_{12}$O$_4$ \\
        15  & LIGC      & lignin carbon                 & C$_{15}$H$_{14}$O$_4$ \\
        16  & LIGCC     & lignin carbon                 & C$_{15}$H$_{14}$O$_4$ \\
        17  & LIGH      & lignin hydrogen               & C$_{22}$H$_{28}$O$_9$ \\
        18  & LIGO      & lignin oxygen                 & C$_{20}$H$_{22}$O$_{10}$ \\
        19  & LIGOH     & lignin hydroxide              & C$_{19}$H$_{22}$O$_8$ \\
        20  & XYHW      & hemicellulose hardwood        & C$_5$H$_8$O$_4$ \\
        \bottomrule
    \end{tabular}
\end{table}

\begin{table}[H]
    \centering
    \caption{Gas species in Ranzi kinetic scheme.}
    \begin{tabular}{@{}clll@{}}
        \toprule
        Item & Abbreviation & Name & Formula \\
        \midrule
        1   & ACAC      & acetic acid               & C$_2$H$_4$O$_2$ \\
        2   & ALD3      & propanal                  & C$_3$H$_6$O \\
        3   & ANISOLE   & anisole                   & C$_7$H$_8$O \\
        4   & C2H4      & ethylene                  & C$_2$H$_4$ \\
        5   & C2H6      & ethane                    & C$_2$H$_6$ \\
        6   & C2H5OH    & ethanol                   & C$_2$H$_6$O \\
        7   & C2H3CHO   & 2-propenal                & C$_3$H$_4$O \\
        8   & C3H6O2    & propanal, 3-hydroxy-      & C$_3$H$_6$O$_2$ \\
        9   & CH2O      & formaldehyde              & CH$_2$O \\
        10  & CH3OH     & methyl alcohol            & CH$_4$O \\
        11  & CH3CHO    & acetaldehyde              & C$_2$H$_4$O \\
        12  & CH4       & methane                   & CH$_4$ \\
        13  & CO        & carbon monoxide           & CO \\
        14  & CO2       & carbon dioxide            & CO$_2$ \\
        15  & COUMARYL  & coumaryl alcohol          & C$_9$H$_{10}$O$_2$ \\
        16  & FE2MACR   & sinapaldehyde             & C$_{11}$H$_{12}$O$_4$ \\
        17  & FURF      & furfural                  & C$_5$H$_4$O$_2$   \\
        18  & GLYOX     & glyoxal                   & C$_2$H$_2$O$_2$ \\
        19  & H2        & hydrogen                  & H$_2$ \\
        20  & H2O       & water                     & H$_2$O \\
        21  & HAA       & hydroxy-acetaldehyde      & C$_2$H$_4$O$_2$ \\
        22  & HCOOH     & formic acid               & CH$_2$O$_2$ \\
        23  & HMFU      & 5-hydroxymethyl-furfural  & C$_6$H$_6$O$_3$ \\
        24  & LVG       & levoglucosan              & C$_6$H$_{10}$O$_5$ \\
        25  & PHENOL    & phenol                    & C$_6$H$_6$O \\
        26  & XYLAN     & xylosan                   & C$_5$H$_8$O$_4$ \\
        \bottomrule
    \end{tabular}
\end{table}
