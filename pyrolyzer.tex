%!TEX root = main.tex

\section{Fluidized bed pyrolyzer}

The biomass thermochemical conversion unit in the NREL 2FBR system is a two-inch diameter bubbling fluidized bed (BFB) reactor that operates at fast pyrolysis conditions. An overview of the pyrolyzer components along with its process flows are shown in Figure \ref{fig:pyrolyzer}. See the following subsections for more information on reactor dimensions, typical operating conditions, feedstock characteristics, bed material properties, and biomass pyrolysis kinetic schemes.

\begin{figure}[H]
    \centering
    \includegraphics[width=\textwidth]{pyrolyzer}
    \caption{Bubbling fluidized bed biomass pyrolysis reactor in the NREL 2FBR system. Reactor components (left) along with inlet and outlet process flows (right).}
    \label{fig:pyrolyzer}
\end{figure}

\subsection{Dimensions and operating conditions}

Dimensions for components in the pyrolyzer are given in Table \ref{tab:dimensions}. The gas distributor contains 18 holes in a triangular pattern. Insulation surrounds the unit while heat jackets extend almost the entire height of the reactor.

Table \ref{tab:operating} gives typical operating conditions for the pyrolyzer. Pressure drop across the distributor is about 80-90 inches of H$_2$O. Nitrogen gas is used to fluidize the bed and feed biomass particles. Experiments are conducted with an initial mass of sand in the bed; sand is not fed into the reactor during operation. Pyrolysis vapors exit directly out the top of the reactor via a straight tube.

\begin{table}[H]
    \centering
    \caption{Dimensions for components of the fluidized bed pyrolysis reactor.}
    \label{tab:dimensions}
    \begin{tabular}{llrll}
        \toprule
        Component & Name & Value & Units & Description \\
        \midrule
        Reactor
            & d$_\textrm{inner}$ & 5.08 & cm & inner diameter \\
            & h$_\textrm{static}$ & 10.16 & cm & static bed height \\
            & h$_\textrm{total}$ & 43.18 & cm & total height of reactor tube \\
        Feed inlet
            & d$_\textrm{inner}$ & x & x & inner diameter \\
            & h$_\textrm{feed}$ & 1.9 & cm & feed height above distributor \\
        Insulation
            & h$_\textrm{insul}$ & 43.18 & cm & insulation height \\
            & th$_\textrm{insul}$ & 10 & cm & thickness \\
        Heat jacket
            & h$_\textrm{jack}$ & 35 & cm & height of jacket \\
            & th$_\textrm{jack}$ & 5 & cm & thickness of jacket \\
        Distributor
            & d$_\textrm{orif}$ & 0.79 & mm & diameter of distributor orifices \\
            & n & 18 & - &number of orifices \\
            & th$_\textrm{dist}$ & 3.17 & mm & thickness of distributor \\
        \bottomrule
    \end{tabular}
\end{table}

\begin{table}[H]
    \centering
    \caption{Typical operating conditions for the fluidized bed pyrolysis reactor.}
    \label{tab:operating}
    \begin{tabular}{llrll}
        \toprule
        Component & Parameter & Value & Units & Description \\
        \midrule
        Reactor
            & p$_\textrm{abs}$ & 101.3 & kPa & absolute pressure in reactor \\
            & p$_\textrm{atm}$ & 81 & kPa & atmospheric pressure \\
            & tk$_\textrm{amb}$ & 300.15 & K & ambient air temperature \\
        Bed
            & p$_\textrm{bed}$ & 115 & kPa & absolute bed pressure \\
            & tk$_\textrm{bed}$ & 773.15 & K & bed temperature \\
        Inlet gas
            & $\Delta$p & 21.17 & kPa & distributor pressure drop \\
            & p$_\textrm{gas}$ & 133 & kPa & absolute inlet gas pressure \\
            & tk$_\textrm{gas}$ & 773.15 & K & inlet gas temperature \\
            & u$_\textrm{gas}$ & 14 & SLM & nitrogen gas flowrate, 0.292 g/s \\
        Secondary gas
            & tk$_\textrm{gas}$ & 298.15 & K & gas temperature \\
            & u$_\textrm{gas}$ & 1.4 & SLM & nitrogen gas flowrate, 0.0292 g/s \\
        Biomass feed
            & $\dot{\textrm{m}}_\textrm{feed}$ & 425 & g/hr & biomass feedrate \\
            & tk$_\textrm{feed}$ & 298.15 & K & biomass temperature \\
        \bottomrule
    \end{tabular}
\end{table}

\subsection{Feedstock properties}

Several woody feedstocks have been tested in the pyrolysis unit; however, loblolly pine is the current focus of ongoing experiments. Properties of loblolly pine from the Wood Handbook are provided in Table XX shown below.

Particle diameters are from the cleanpine-particle-size.xlsx spreadsheet provided by NREL.

Loblolly pine heat capacity of pine calculated from $0.1031 + 0.003867 T$ kJ/(kg K). Particle size 10th percentile 133.2 um, particle size 50th percentile 534.4 um, and particle size 90th percentile 1057.4 um. Surface weighted mean size 248.2 um and volume weighted mean size 573.9 um. Ultimate analysize values on a wet basis taken from Iisa 2017 paper are C 49.6\%, H 6.3\%, O 43.5 \%, N 0.1\%, S less than 0.1\%, ash 0.3\%, and water content 2.3\%.

\begin{table}[H]
    \centering
    \caption{Loblolly pine feedstock properties for NREL 2FBR pyrolyzer. Heat capacity is a function of temperature.}
    \label{tab:feedstock}
    \begin{tabular}{lrll}
        \toprule
        Property & Value & Units & Description \\
        \midrule
        C & 50.6 & \% & ultimate analysis carbon \\
        H & 6.6 & \% & ultimate analysis hydrogen \\
        O & 42.35 & \% & ultimate analysis oxygen \\
        N & 0.03 & \% & ultimate analysis nitrogen \\
        S & 0.02 & \% & ultimate analysis sulfur \\
        ash & 0.4 & \% & ultimate analysis ash \\
        VM & 85.2 & \% & proximate analysis volatile matter \\
        FC & 11.5 & \% & proximate analysis fixed carbon \\
        MC & 2.9 & \% & proximate analysis moisture content \\
        ash & 0.4 & \% & proximate analysis ash \\
        $\rho$ & 540 & kg/m$^3$ & particle density \\
        C$_\textrm{p}$ & $0.1031 + 0.003867\;\textrm{T}$ & kJ/kgK & heat capacity \\
        k & 0.12 & W/mK & thermal conductivity \\
        \bottomrule
    \end{tabular}
\end{table}

\subsection{Bed properties}

Sand properties for the pyrolyzer bed material are presented in Table \ref{tab:sand-properties}. Particle size as sieved was reported as 0.3 to 0.5 mm.

\begin{table}[H]
    \centering
    \caption{Bed properties for fluidized sand particles in the pyrolyzer.}
    \label{tab:sand-properties}
    \begin{tabular}{lrll}
        \toprule
        Property & Value & Units & Description \\
        \midrule
        d$_\textrm{p}$ & 0.0003 & m & particle diameter \\
        $\epsilon$ & 0.76 & - & average emissivity of particles \\
        m$_\textrm{bed}$ & 330 & g & mass of bed material \\
        $\phi$ & 0.86 & - & particle sphericity \\
        $\rho_\textrm{app}$ & 2500 & kg/m$^3$ & apparent density \\
        $\rho_\textrm{bulk}$ & 1510 & kg/m$^3$ & bulk density \\
        \bottomrule
    \end{tabular}
\end{table}

\subsection{Insulation and heat jacket properties}

The BFB pyrolyzer is insulated along the height of the reactor.

\begin{table}[H]
    \centering
    \caption{Insulation properties for the pyrolysis reactor.}
    \label{tab:insulation}
    \begin{tabular}{lrll}
        \toprule
        Property & Value & Units & Description \\
        \midrule
        $\epsilon$ & 0.92 & - & emissivity of insulation surface \\
        k & 0.0022 & W/mK & thermal conductivity \\
        \bottomrule
    \end{tabular}
\end{table}

\begin{table}[H]
    \centering
    \caption{Heat jacket properties for the pyrolysis reactor.}
    \label{tab:heatjacket}
    \begin{tabular}{lrll}
        \toprule
        Property & Value & Units & Description \\
        \midrule
        q & 0.38 & kW & heat transfer to interior \\
        \bottomrule
    \end{tabular}
\end{table}

\subsection{Biomass pyrolysis kinetics}

The Ranzi et al. kinetic scheme for biomass fast pyrolysis is described in this section.

\begin{table}[H]
    \centering
    \caption{Solid species in Ranzi kinetic scheme.}
    \begin{tabular}{@{}clll@{}}
        \toprule
        Item & Abbreviation & Name & Formula \\
        \midrule
        1   & CELL      & cellulose                     & C$_6$H$_{10}$O$_5$ \\
        2   & CELLA     & activated cellulose           & C$_6$H$_{10}$O$_5$ \\
        3   & CHAR      & char                          & C \\
        4   & GCO       & metaplastic carbon monoxide   & CO \\
        5   & GCO2      & metaplastic carbon dioxide    & CO$_2$ \\
        6   & GCH4      & metaplastic methane           & CH$_4$ \\
        7   & GC2H4     & metaplastic ethylene          & C$_2$H$_4$ \\
        8   & GCH3OH    & metaplastic methyl alcohol    & CH$_4$O \\
        9   & GCOH2     & metaplastic formaldehyde      & CH$_2$O \\
        10  & GH2       & metaplastic hydrogen          & H$_2$ \\
        11  & GMSW      & hemicellulose softwood        & C$_5$H$_8$O$_4$ \\
        12  & HCE1      & hemicellulose                 & C$_5$H$_8$O$_4$ \\
        13  & HCE2      & hemicellulose                 & C$_5$H$_8$O$_4$ \\
        9   & HMWL      & heavy molecular weight lignin & C$_{24}$H$_{28}$O$_4$ \\
        14  & LIG       & lignin                        & C$_{11}$H$_{12}$O$_4$ \\
        15  & LIGC      & lignin carbon                 & C$_{15}$H$_{14}$O$_4$ \\
        16  & LIGCC     & lignin carbon                 & C$_{15}$H$_{14}$O$_4$ \\
        17  & LIGH      & lignin hydrogen               & C$_{22}$H$_{28}$O$_9$ \\
        18  & LIGO      & lignin oxygen                 & C$_{20}$H$_{22}$O$_{10}$ \\
        19  & LIGOH     & lignin hydroxide              & C$_{19}$H$_{22}$O$_8$ \\
        20  & XYHW      & hemicellulose hardwood        & C$_5$H$_8$O$_4$ \\
        \bottomrule
    \end{tabular}
\end{table}

\begin{table}[H]
    \centering
    \caption{Gas species in Ranzi kinetic scheme.}
    \begin{tabular}{@{}clll@{}}
        \toprule
        Item & Abbreviation & Name & Formula \\
        \midrule
        1   & ACAC      & acetic acid               & C$_2$H$_4$O$_2$ \\
        2   & ALD3      & propanal                  & C$_3$H$_6$O \\
        3   & ANISOLE   & anisole                   & C$_7$H$_8$O \\
        4   & C2H4      & ethylene                  & C$_2$H$_4$ \\
        5   & C2H6      & ethane                    & C$_2$H$_6$ \\
        6   & C2H5OH    & ethanol                   & C$_2$H$_6$O \\
        7   & C2H3CHO   & 2-propenal                & C$_3$H$_4$O \\
        8   & C3H6O2    & propanal, 3-hydroxy-      & C$_3$H$_6$O$_2$ \\
        9   & CH2O      & formaldehyde              & CH$_2$O \\
        10  & CH3OH     & methyl alcohol            & CH$_4$O \\
        11  & CH3CHO    & acetaldehyde              & C$_2$H$_4$O \\
        12  & CH4       & methane                   & CH$_4$ \\
        13  & CO        & carbon monoxide           & CO \\
        14  & CO2       & carbon dioxide            & CO$_2$ \\
        15  & COUMARYL  & coumaryl alcohol          & C$_9$H$_{10}$O$_2$ \\
        16  & FE2MACR   & sinapaldehyde             & C$_{11}$H$_{12}$O$_4$ \\
        17  & FURF      & furfural                  & C$_5$H$_4$O$_2$   \\
        18  & GLYOX     & glyoxal                   & C$_2$H$_2$O$_2$ \\
        19  & H2        & hydrogen                  & H$_2$ \\
        20  & H2O       & water                     & H$_2$O \\
        21  & HAA       & hydroxy-acetaldehyde      & C$_2$H$_4$O$_2$ \\
        22  & HCOOH     & formic acid               & CH$_2$O$_2$ \\
        23  & HMFU      & 5-hydroxymethyl-furfural  & C$_6$H$_6$O$_3$ \\
        24  & LVG       & levoglucosan              & C$_6$H$_{10}$O$_5$ \\
        25  & PHENOL    & phenol                    & C$_6$H$_6$O \\
        26  & XYLAN     & xylosan                   & C$_5$H$_8$O$_4$ \\
        \bottomrule
    \end{tabular}
\end{table}
